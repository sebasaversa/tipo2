\section{Filtro de tiles}
\subsection{Enunciado}

 Programar el filtro \textit{tiles} en lenguaje C y luego en ASM haciendo uso de las instrucciones vectoriales (\textbf{SSE}).

\vspace*{0.3cm} \noindent
\textbf{Experimento 1 - análisis el código generado}

Utilizar la herramienta \verb|objdump| para verificar como el compilador de C deja ensamblado el código C. Como es el código generado, ¿cómo se manipulan las variables locales?¿le parece que ese código generado podría optimizarse?

\vspace*{0.3cm} \noindent
\textbf{Experimento 2 - optimizaciones del compilador}

Compile el código de C con optimizaciones del compilador, por ejemplo, pasando el flag \verb|-O1|\footnote{agregando este flag a \texttt{CCFLAGS64} en el makefile}. 
¿Qué optimizaciones realizó el compilador?
¿Qué otros flags de optimización brinda el compilador?
¿Para qué sirven?


\vspace*{0.3cm} \noindent
\textbf{Experimento 3 - secuencial vs. vectorial}

	Realice una medición de las diferencias de performance entre las versiones
	de C y ASM (el primero con -O1, -O2 y -O3).\\
	¿Como realizó la medición?¿Cómo sabe que su medición es una buena medida?¿Cómo afecta a la medición la existencia de \emph{outliers}\footnote{en español, valor atípico: \url{http://es.wikipedia.org/wiki/Valor_atípico}}?¿De qué manera puede minimizar su impacto?¿Qué resultados obtiene si mientras corre los tests ejecuta otras aplicaciones que utilicen al máximo la CPU? 
	Realizar un análisis \textbf{riguroso} de los resultados y acompañar con un gráfico que presente estas diferencias.


\vspace*{0.3cm} \noindent
\textbf{Experimento 4 - cpu vs. bus de memoria}

	Se desea conocer cual es el mayor limitante a la performance de este filtro en su versión ASM.

	¿Cuál es el factor que limita la performance en este caso? En caso de que el limitante
	fuera la intensidad de cómputo, entonces podrían agregarse instrucciones que realicen
	accesos a memoria y la performance casi no debería sufrir. La inversa puede aplicarse
	si el limitante es la cantidad de accesos a memoria.
	
	Realizar un experimento, agregando múltiples instrucciones de un mismo tipo y realizar un análisis
	del resultado. Acompañar con un gráfico.

\vspace*{0.3cm} \noindent
\textbf{Experimento 5 (\textit{opcional}) - secuencial vs. vectorial (parte II)}

	Si vemos a los pixeles como una tira muy larga de bytes, este filtro en
	realidad no requiere ningún procesamiento de datos en paralelo. Esto podría
	significar que la velocidad del filtro de C puede aumentarse hasta casi
	alcanzar la del de ASM. ¿ocurre esto?
	
	Modificar el filtro para que en vez de acceder a los bytes de a uno a la vez
	se accedan como tiras de 64 bits y analizar la performance.
	
\newpage
\subsection{Implementaci\'on en C}

La implementaci\'on en C del ejercicio tiles es tomar un rect\'angulo de la imagen original y repartirla por la imagen destino. Se resuelve con dos ciclos for anidados usando la forma dada en el enuciado.

\subsection{Implementaci\'on en assembler}
Para la resoluci\'on en assembler creamos los dos ciclos anidados con los registros de prop\'osito general. AL comienzo del segundo siclo se hace la copia.
\begin{codesnippet}
\begin{verbatim}
	MOVDQU XMM0, [RDI]
	MOVDQU [RSI], XMM0
\end{verbatim}
\end{codesnippet}
Luego, para ciclar y recorrer toda la matriz se modifican los valores de RSI y RDI cuando es necesario.
\subsection{Resultado de los experimentos}

\vspace*{0.3cm} \noindent
\textbf{Experimento 1 - análisis el código generado}
El compilador genera más variables locales que las usadas en nuestro c\'odigo assembler. En ese sentido, estamos realizando una mejora al c\'odigo de memoria y velocidad


\vspace*{0.3cm} \noindent
\textbf{Experimento 2 - optimizaciones del compilador}
El compilador ofrece los flags -O1 a -03 para distintos niveles de optimizaci\'on, asi como los flags para indicar si es assembler de intel, amd, etc. Esto ultimo hace m\'as espec\'ifico el codigo assembler y correr\'ia mejor en una maquina con el microprocesador seleccionado,aunque no correr\'ia en uno de otra marca o de una familia de microprocesadores anterior a la elegida.
