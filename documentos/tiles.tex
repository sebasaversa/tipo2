\section{Filtro de tiles}
\subsection{Enunciado}

 Programar el filtro \textit{tiles} en lenguaje C y luego en ASM haciendo uso de las instrucciones vectoriales (\textbf{SSE}).

\vspace*{0.3cm} \noindent
\textbf{Experimento 1 - análisis el código generado}

Utilizar la herramienta \verb|objdump| para verificar como el compilador de C deja ensamblado el código C. Como es el código generado, ¿cómo se manipulan las variables locales?¿le parece que ese código generado podría optimizarse?

\vspace*{0.3cm} \noindent
\textbf{Experimento 2 - optimizaciones del compilador}

Compile el código de C con optimizaciones del compilador, por ejemplo, pasando el flag \verb|-O1|\footnote{agregando este flag a \texttt{CCFLAGS64} en el makefile}. 
¿Qué optimizaciones realizó el compilador?
¿Qué otros flags de optimización brinda el compilador?
¿Para qué sirven?


\vspace*{0.3cm} \noindent
\textbf{Experimento 3 - secuencial vs. vectorial}

	Realice una medición de las diferencias de performance entre las versiones
	de C y ASM (el primero con -O1, -O2 y -O3).\\
	¿Como realizó la medición?¿Cómo sabe que su medición es una buena medida?¿Cómo afecta a la medición la existencia de \emph{outliers}\footnote{en español, valor atípico: \url{http://es.wikipedia.org/wiki/Valor_atípico}}?¿De qué manera puede minimizar su impacto?¿Qué resultados obtiene si mientras corre los tests ejecuta otras aplicaciones que utilicen al máximo la CPU? 
	Realizar un análisis \textbf{riguroso} de los resultados y acompañar con un gráfico que presente estas diferencias.


\vspace*{0.3cm} \noindent
\textbf{Experimento 4 - cpu vs. bus de memoria}

	Se desea conocer cual es el mayor limitante a la performance de este filtro en su versión ASM.

	¿Cuál es el factor que limita la performance en este caso? En caso de que el limitante
	fuera la intensidad de cómputo, entonces podrían agregarse instrucciones que realicen
	accesos a memoria y la performance casi no debería sufrir. La inversa puede aplicarse
	si el limitante es la cantidad de accesos a memoria.
	
	Realizar un experimento, agregando múltiples instrucciones de un mismo tipo y realizar un análisis
	del resultado. Acompañar con un gráfico.

\vspace*{0.3cm} \noindent
\textbf{Experimento 5 (\textit{opcional}) - secuencial vs. vectorial (parte II)}

	Si vemos a los pixeles como una tira muy larga de bytes, este filtro en
	realidad no requiere ningún procesamiento de datos en paralelo. Esto podría
	significar que la velocidad del filtro de C puede aumentarse hasta casi
	alcanzar la del de ASM. ¿ocurre esto?
	
	Modificar el filtro para que en vez de acceder a los bytes de a uno a la vez
	se accedan como tiras de 64 bits y analizar la performance.
	
\newpage

\subsection{An\'alisis previo}
El filtro de tiles define un rect\'angulo en una imagen en color. La funci\'on debe replicar dicho sector de la im\'agen en el resto de la imagen original.
Para esto, se nos ocurrieron tres formas de encararlo: con la f\'ormula dada en el enunciado (versi\'on C), guardando el primer elemento del rect\'angulo a copiar y guardando un puntero a ese primer elemento.

Formula:

i y j representan fila y columna respectivamente.

dst(i,j) = src [(i mod tamy)+offsety][(j mod tamx)+offsetx]

\subsection{Implementaci\'on en C}

La implementaci\'on en C del ejercicio tiles es tomar un rect\'angulo de la imagen original y repartirla por la imagen destino. Se resuelve con dos ciclos for anidados usando la forma dada en el enuciado.

\subsection{Implementaci\'on en assembler}
Para la resoluci\'on en assembler creamos los dos ciclos anidados con los registros de prop\'osito general. AL comienzo del segundo ciclo se hace la copia.
\begin{codesnippet}
\begin{verbatim}
	MOVDQU XMM0, [RDI]
	MOVDQU [RSI], XMM0
\end{verbatim}
\end{codesnippet}
Luego, para ciclar y recorrer toda la matriz se modifican los valores de RSI y RDI cuando es necesario.
\subsection{Resultado de los experimentos}

\vspace*{0.3cm} \noindent
\textbf{Experimento 1 - análisis el código generado}
\subsubsection{Desensamblado de tiles.o:}
objdump -M intel -d tiles_c.o

tiles_c.o:     formato del fichero elf64-x86-64

Desensamblado de la sección .text:

\begin{codesnippet}
\begin{verbatim}
       push   rbp
       mov    rbp,rsp
       push   r13
       push   r12
       mov    QWORD PTR [rbp-0x58],rdi
       mov    QWORD PTR [rbp-0x60],rsi
       mov    DWORD PTR [rbp-0x64],edx
       mov    DWORD PTR [rbp-0x68],ecx
       mov    DWORD PTR [rbp-0x6c],r8d
       mov    DWORD PTR [rbp-0x70],r9d
       mov    ecx,DWORD PTR [rbp-0x6c]
       movsxd rax,ecx
       sub    rax,0x1
       mov    QWORD PTR [rbp-0x40],rax
       movsxd rax,ecx
       mov    r12,rax
       mov    r13d,0x0
       mov    rax,QWORD PTR [rbp-0x58]
       mov    QWORD PTR [rbp-0x38],rax
       mov    esi,DWORD PTR [rbp-0x70]
       movsxd rax,esi
       sub    rax,0x1
       mov    QWORD PTR [rbp-0x30],rax
       movsxd rax,esi
       mov    r10,rax
       mov    r11d,0x0
       mov    rax,QWORD PTR [rbp-0x60]
       mov    QWORD PTR [rbp-0x28],rax
       mov    DWORD PTR [rbp-0x48],0x0
       jmp    119 <tiles_c+0x119>
 6e    mov    DWORD PTR [rbp-0x44],0x0
       jmp    109 <tiles_c+0x109>
 7a    mov    eax,DWORD PTR [rbp-0x48]
       movsxd rdx,eax
       movsxd rax,esi
       imul   rdx,rax
       mov    rax,QWORD PTR [rbp-0x28]
       lea    rdi,[rdx+rax*1]
       mov    edx,DWORD PTR [rbp-0x44]
       mov    eax,edx
       add    eax,eax
       add    eax,edx
       cdqe   
       add    rax,rdi
       mov    QWORD PTR [rbp-0x20],rax
       mov    eax,DWORD PTR [rbp-0x48]
       cdq    
       idiv   DWORD PTR [rbp+0x18]
       mov    eax,DWORD PTR [rbp+0x28]
       add    eax,edx
       movsxd rdx,eax
       movsxd rax,ecx
       imul   rdx,rax
       mov    rax,QWORD PTR [rbp-0x38]
       lea    r8,[rdx+rax*1]
       mov    edx,DWORD PTR [rbp-0x44]
       mov    eax,edx
       add    eax,eax
       lea    edi,[rax+rdx*1]
       mov    edx,DWORD PTR [rbp+0x10]
\end{verbatim}
\end{codesnippet}
\begin{codesnippet}
\begin{verbatim}
       mov    eax,edx
       add    eax,eax
       lea    r9d,[rax+rdx*1]
       mov    eax,edi
       cdq    
       idiv   r9d
       mov    edi,edx
       mov    edx,DWORD PTR [rbp+0x20]
       mov    eax,edx
       add    eax,eax
       add    eax,edx
       add    eax,edi
       cdqe   
       add    rax,r8
       mov    QWORD PTR [rbp-0x18],rax
       mov    rax,QWORD PTR [rbp-0x20]
       mov    rdx,QWORD PTR [rbp-0x18]
       movzx  edi,WORD PTR [rdx]
       mov    WORD PTR [rax],di
       movzx  edx,BYTE PTR [rdx+0x2]
       mov    BYTE PTR [rax+0x2],dl
       add    DWORD PTR [rbp-0x44],0x1
109    mov    eax,DWORD PTR [rbp-0x44]
       cmp    eax,DWORD PTR [rbp-0x64]
       jl     7a <tiles_c+0x7a>
       add    DWORD PTR [rbp-0x48],0x1
119    mov    eax,DWORD PTR [rbp-0x48]
       cmp    eax,DWORD PTR [rbp-0x68]
       jl     6e <tiles_c+0x6e>
       pop    r12
       pop    r13
       pop    rbp
       ret   
\end{verbatim}
\end{codesnippet}
\begin{codesnippet}
\begin{verbatim}
tiles_asm:
	PUSH RBP	
	MOV RBP, RSP
	XOR R10, R10
	MOV R10D, [RBP+32] ; coloco offsetX
	XOR R11, R11
	.muevoX:
	CMP R11, R10
	JE .bajoY
	LEA RDI, [RDI+3] ;muevo 3 bytes el puntero de la matriz src
	INC R11
	JMP .muevoX
	.bajoY:
	XOR R10, R10
	MOV R10D, [RBP+40] ;coloco offsetY
	XOR R11, R11
	.bajoY2:
	CMP R11, R10
	JE .inicio
	LEA RDI, [RDI+R8] ;aumento en row size bytes el puntero ( bajo una fila)
	INC R11
	JMP .bajoY2
	.inicio:
	MOV R13, RDI
	XOR R11, R11
	XOR R12, R12
	LEA R13, [RDI]
	LEA R12, [RDI]
	XOR R14, R14	; R14: i
	.for1:
		CMP R14, RCX
		JGE .endfor1
		XOR R10, R10
		XOR R15, R15	;R15: j
		MOV RAX, RDI
		.for2:
			CMP R15, RDX
			JGE .endfor2
			CMP R10D, [RBP+16]
			JB .mePaseTile
			XOR R10, R10
			LEA RDI, [R13]
			JMP .for2
			.mePaseTile:
			MOV R11, R10
			ADD R11, 6
			CMP R11D, [RBP+16]
			JBE .sigo
			INC R10
			INC R15
			LEA RDI, [RDI - 13]
			LEA RSI, [RSI - 13]
			MOVDQU XMM0, [RDI]
\end{verbatim}
\end{codesnippet}
\begin{codesnippet}
\begin{verbatim}
			MOVDQU [RSI], XMM0
			LEA RDI, [RDI + 16]
			LEA RSI, [RSI + 16]
			JMP .for2
			.sigo:
			MOV R11, R15
			ADD R11, 6
			CMP R11, RDX
			JBE .sigo2
			INC R10
			INC R15
			;LEA RDI, [RDI - 13]
			LEA RSI, [RSI - 13]
			MOVDQU XMM0, [RDI]
			MOVDQU XMM1, [RSI] ;ME GUARDO LOS ANTERIORES 13 BYTES DE RSI
			PSLLDQ XMM1, 3
			PSRLDQ XMM1, 3
			PSLLDQ XMM0, 13
			PADDB XMM0, XMM1
			MOVDQU [RSI], XMM0s
			LEA RDI, [RDI + 3]
			LEA RSI, [RSI + 16]
			JMP .for2
			.sigo2:
			MOVDQU XMM0, [RDI]
			MOVDQU [RSI], XMM0
			ADD R10, 5
			ADD R15, 5
			LEA RDI, [RDI + 15]
			LEA RSI, [RSI + 15]
			JMP .for2
			.endfor2:
			INC R14
			XOR RAX, RAX
			XOR R15, R15
			MOV EAX, [RBP + 24] ;TAM Y
			MOV R15, R14 ; FILAS QUE RECORRI
			.cicloDiv:
			CMP R15, RAX
			JB .muevoRecuadro
			JE .inicioTiles
			SUB R15, RAX
			CMP R15, RAX
			JG .cicloDiv
			JB .muevoRecuadro
			.inicioTiles:
			LEA RDI, [R12]
			LEA R13, [RDI]
			JMP .muevoDST
			.muevoRecuadro:
			LEA RDI, [R13 + R8]
			LEA R13, [RDI]
\end{verbatim}
\end{codesnippet}
\begin{codesnippet}
\begin{verbatim}
			.muevoDST:
			MOV R15, RDX
			MOV RAX, RDX
			ADD RAX, RAX
			ADD R15, RAX
			MOV RAX, R9
			SUB RAX, R15
			LEA RSI, [RSI + RAX]
			JMP .for1
	.endfor1:
	POP RBP
	RET
\end{verbatim}
\end{codesnippet}


\vspace*{0.3cm} \noindent
\textbf{Experimento 2 - optimizaciones del compilador}
El compilador ofrece los flags -O1 a -03 para distintos niveles de optimizaci\'on, asi como los flags para indicar si es assembler de intel, amd, etc. 
Esto ultimo hace m\'as espec\'ifico el codigo assembler y correr\'ia mejor en una maquina con el microprocesador seleccionado,aunque no correr\'ia en uno de otra marca o de una familia de microprocesadores anterior a la elegida.

\vspace*{0.3cm} \noindent
\textbf{Experimento 3 - secuencial vs. vectorial}

\vspace*{0.3cm} \noindent
\textbf{Experimento 4 - cpu vs. bus de memoria}

\vspace*{0.3cm} \noindent
\textbf{Experimento 5 (\textit{opcional}) - secuencial vs. vectorial (parte II)}
