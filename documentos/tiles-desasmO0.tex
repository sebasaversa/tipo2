\subsubsection{Desensamblado de tiles.o:}
objdump -M intel -d tiles_c.o

tiles_c.o:     formato del fichero elf64-x86-64

Desensamblado de la sección .text:

\begin{codesnippet}
\begin{verbatim}
       push   rbp
       mov    rbp,rsp
       push   r13
       push   r12
       mov    QWORD PTR [rbp-0x58],rdi	;en este caso PTR no hace diferencia. Es lo mismo a hacer mov QWORD [rbp-0x58], rdi
       mov    QWORD PTR [rbp-0x60],rsi	;sabemos que estos mov los hace de más
       mov    DWORD PTR [rbp-0x64],edx
       mov    DWORD PTR [rbp-0x68],ecx
       mov    DWORD PTR [rbp-0x6c],r8d
       mov    DWORD PTR [rbp-0x70],r9d
       mov    ecx,DWORD PTR [rbp-0x6c]	;mov ecx, r8d
       
		;hay q encontrar los 2 for
       
       movsxd rax,ecx
       sub    rax,0x1
       mov    QWORD PTR [rbp-0x40],rax
       movsxd rax,ecx
       mov    r12,rax
       mov    r13d,0x0
       mov    rax,QWORD PTR [rbp-0x58]
       mov    QWORD PTR [rbp-0x38],rax
       mov    esi,DWORD PTR [rbp-0x70]
       movsxd rax,esi
       sub    rax,0x1
       mov    QWORD PTR [rbp-0x30],rax
       movsxd rax,esi
       mov    r10,rax
       mov    r11d,0x0
       mov    rax,QWORD PTR [rbp-0x60]
       mov    QWORD PTR [rbp-0x28],rax
       mov    DWORD PTR [rbp-0x48],0x0
       jmp    119 <tiles_c+0x119>
 6e    mov    DWORD PTR [rbp-0x44],0x0
       jmp    109 <tiles_c+0x109>
 7a    mov    eax,DWORD PTR [rbp-0x48]
       movsxd rdx,eax
       movsxd rax,esi
       imul   rdx,rax
       mov    rax,QWORD PTR [rbp-0x28]
       lea    rdi,[rdx+rax*1]
       mov    edx,DWORD PTR [rbp-0x44]
       mov    eax,edx
       add    eax,eax
       add    eax,edx
       cdqe   
       add    rax,rdi
       mov    QWORD PTR [rbp-0x20],rax
       mov    eax,DWORD PTR [rbp-0x48]
       cdq    
       idiv   DWORD PTR [rbp+0x18]
       mov    eax,DWORD PTR [rbp+0x28]
       add    eax,edx
       movsxd rdx,eax
       movsxd rax,ecx
       imul   rdx,rax
       mov    rax,QWORD PTR [rbp-0x38]
       lea    r8,[rdx+rax*1]
       mov    edx,DWORD PTR [rbp-0x44]
       mov    eax,edx
       add    eax,eax
       lea    edi,[rax+rdx*1]
       mov    edx,DWORD PTR [rbp+0x10]
\end{verbatim}
\end{codesnippet}
\begin{codesnippet}
\begin{verbatim}
       mov    eax,edx
       add    eax,eax
       lea    r9d,[rax+rdx*1]
       mov    eax,edi
       cdq    
       idiv   r9d
       mov    edi,edx
       mov    edx,DWORD PTR [rbp+0x20]
       mov    eax,edx
       add    eax,eax
       add    eax,edx
       add    eax,edi
       cdqe   
       add    rax,r8
       mov    QWORD PTR [rbp-0x18],rax
       mov    rax,QWORD PTR [rbp-0x20]
       mov    rdx,QWORD PTR [rbp-0x18]
       movzx  edi,WORD PTR [rdx]
       mov    WORD PTR [rax],di
       movzx  edx,BYTE PTR [rdx+0x2]
       mov    BYTE PTR [rax+0x2],dl
       add    DWORD PTR [rbp-0x44],0x1
109    mov    eax,DWORD PTR [rbp-0x44]
       cmp    eax,DWORD PTR [rbp-0x64]
       jl     7a <tiles_c+0x7a>
       add    DWORD PTR [rbp-0x48],0x1
119    mov    eax,DWORD PTR [rbp-0x48]
       cmp    eax,DWORD PTR [rbp-0x68]
       jl     6e <tiles_c+0x6e>
       pop    r12
       pop    r13
       pop    rbp
       ret   
\end{verbatim}
\end{codesnippet}
\begin{codesnippet}
\begin{verbatim}
tiles_asm:
	PUSH RBP	
	MOV RBP, RSP
	XOR R10, R10
	MOV R10D, [RBP+32] ; coloco offsetX
	XOR R11, R11
	.muevoX:
	CMP R11, R10
	JE .bajoY
	LEA RDI, [RDI+3] ;muevo 3 bytes el puntero de la matriz src
	INC R11
	JMP .muevoX
	.bajoY:
	XOR R10, R10
	MOV R10D, [RBP+40] ;coloco offsetY
	XOR R11, R11
	.bajoY2:
	CMP R11, R10
	JE .inicio
	LEA RDI, [RDI+R8] ;aumento en row size bytes el puntero ( bajo una fila)
	INC R11
	JMP .bajoY2
	.inicio:
	MOV R13, RDI
	XOR R11, R11
	XOR R12, R12
	LEA R13, [RDI]
	LEA R12, [RDI]
	XOR R14, R14	; R14: i
	.for1:
		CMP R14, RCX
		JGE .endfor1
		XOR R10, R10
		XOR R15, R15	;R15: j
		MOV RAX, RDI
		.for2:
			CMP R15, RDX
			JGE .endfor2
			CMP R10D, [RBP+16]
			JB .mePaseTile
			XOR R10, R10
			LEA RDI, [R13]
			JMP .for2
			.mePaseTile:
			MOV R11, R10
			ADD R11, 6
			CMP R11D, [RBP+16]
			JBE .sigo
			INC R10
			INC R15
			LEA RDI, [RDI - 13]
			LEA RSI, [RSI - 13]
			MOVDQU XMM0, [RDI]
\end{verbatim}
\end{codesnippet}
\begin{codesnippet}
\begin{verbatim}
			MOVDQU [RSI], XMM0
			LEA RDI, [RDI + 16]
			LEA RSI, [RSI + 16]
			JMP .for2
			.sigo:
			MOV R11, R15
			ADD R11, 6
			CMP R11, RDX
			JBE .sigo2
			INC R10
			INC R15
			;LEA RDI, [RDI - 13]
			LEA RSI, [RSI - 13]
			MOVDQU XMM0, [RDI]
			MOVDQU XMM1, [RSI] ;ME GUARDO LOS ANTERIORES 13 BYTES DE RSI
			PSLLDQ XMM1, 3
			PSRLDQ XMM1, 3
			PSLLDQ XMM0, 13
			PADDB XMM0, XMM1
			MOVDQU [RSI], XMM0s
			LEA RDI, [RDI + 3]
			LEA RSI, [RSI + 16]
			JMP .for2
			.sigo2:
			MOVDQU XMM0, [RDI]
			MOVDQU [RSI], XMM0
			ADD R10, 5
			ADD R15, 5
			LEA RDI, [RDI + 15]
			LEA RSI, [RSI + 15]
			JMP .for2
			.endfor2:
			INC R14
			XOR RAX, RAX
			XOR R15, R15
			MOV EAX, [RBP + 24] ;TAM Y
			MOV R15, R14 ; FILAS QUE RECORRI
			.cicloDiv:
			CMP R15, RAX
			JB .muevoRecuadro
			JE .inicioTiles
			SUB R15, RAX
			CMP R15, RAX
			JG .cicloDiv
			JB .muevoRecuadro
			.inicioTiles:
			LEA RDI, [R12]
			LEA R13, [RDI]
			JMP .muevoDST
			.muevoRecuadro:
			LEA RDI, [R13 + R8]
			LEA R13, [RDI]
\end{verbatim}
\end{codesnippet}
\begin{codesnippet}
\begin{verbatim}
			.muevoDST:
			MOV R15, RDX
			MOV RAX, RDX
			ADD RAX, RAX
			ADD R15, RAX
			MOV RAX, R9
			SUB RAX, R15
			LEA RSI, [RSI + RAX]
			JMP .for1
	.endfor1:
	POP RBP
	RET
\end{verbatim}
\end{codesnippet}
