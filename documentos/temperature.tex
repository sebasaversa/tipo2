\section{Filtro Temperature}
\subsection{Enunciado}

\subsection*{Filtro \textit{Temperature}}

  Programar el filtro \textit{Temperature} en lenguaje C y en en ASM haciendo uso de 
  las instrucciones vectoriales (\textbf{SSE}).


\vspace*{0.3cm} \noindent
\textbf{Experimento 1}

  Analizar cuales son las diferencias de performace entre las versiones de C y ASM. 
  Realizar gráficos que representen estas diferencias.
  
\subsection{An\'alisis previo}

El filtro temperature nuevamente presenta tres partes: el ciclo, la elecci\'on de c\'omo modificar los pixeles y la modificaci\'on propiamente dicha.

En \'este caso, la eleccin de los pixeles sale del promedio de los elementos RGB, y la modificaci\'on se realiza en base al promedio.

\subsection{Implementaci\'on en C}

Decidimos no separar la comparacion del pixel de su modificaci\'on porque cada pixel se modifica usando el valor del promedio de los valores RGB. El resto de la implementaci\'on es el ciclo del main que recorre toda la imagen.

\begin{codesnippet}
\begin{verbatim}
void temperatura(rgb_t* p_d, unsigned int r, unsigned int g, unsigned int b)
{
    unsigned int prom = (r + g + b) / 3;

    if 		(prom < 32)		{p_d->r = 0; 				p_d->g = 0; 				p_d->b = 128 + 4*prom;}
    else if (prom < 96)		{p_d->r = 0; 				p_d->g = -128 + 4*prom; 	p_d->b = 255;}
    else if (prom < 160)	{p_d->r = -384 + 4*prom;	p_d->g = 255; 				p_d->b = 639 - 4*prom;}
    else if (prom < 224)	{p_d->r = 255; 				p_d->g = 895 - 4*prom;	 	p_d->b = 0;}
    else 					{p_d->r = 1151 - 4*prom;	p_d->g = 0; 				p_d->b = 0;}
}
\end{verbatim}
\end{codesnippet}
\subsection{Implementaci\'on en assembler}

En este algoritmo tenemos ciertas partes muy parecidas a \textit{popart_asm}, como la sumatoria de los elementos RGB con \verb|PSHUFB| 
(para luego dividir por tres y obtener el promedio).

Sin embargo en este caso estas partes fueron encaradas de forma distinta. \\
Si comparamos los 2 algoritmos que realizan \verb|PSHUFB|, el algoritmo de la funci\'on \textit{popart} trabaja de a 4 pixels y la funci\'on 
\textit{temp_aux} de a 2 pixels.

Pero por ciclo, \textit{temperature_asm} procesa 5 pixeles en vez de los 4 que procesa \textit{popart_asm} por ciclo
\subsubsection{tempe:}
\begin{codesnippet}
\begin{verbatim}
tempe:
    PUSH RBP
    MOV RBP, RSP
        ;XMM0: R|G|B|R|G|B|R|G|B|R|G|B|R|G|B|R coloco los primeros 16bytes de la imagen en XMM1
        ;XMM0: 0|1|2|3|4|5|6|7|8|9|10|11|12|13|14|15
        ;llamare al resultado: R0|R1|R2|R3|R4|R5|R6|R7|R8|R9|R10|R11|R12|R13|R14|-
        MOVDQU XMM8, XMM0
        CALL temp_aux
        MOVDQU XMM9, XMM0
        PSHUFB XMM8, [moverPixeles]  ; XMM8: 6 |7 |8 |9 |10|11|12|13|14|15|- |- |- |- |- |-
        PSHUFB XMM9, [moverPixeles2] ; XMM9: - |- |- |- |- |- |- |- |- |R0|R1|R2|R3|R4|R5|-
        MOVDQU XMM0, XMM8	
        CALL temp_aux
        MOVDQA XMM13, XMM0
        PSHUFB XMM13, [moverPixeles3]; xmm13: - |- |- |R6|R7|R8|R9|R10|R11|- |- |- |- | -|- |-

        PSHUFB XMM8, [moverPixeles1] ; xmm8:  12|13|14|- |- |- |- |- |- |- |- |- |- |- |- |-
        MOVDQU XMM0, XMM8
        CALL temp_aux                ; xmm0:  R12|R13|R14|basura
        PSHUFB XMM0, [unPixel]       ; xmm0:  R12|R13|R14|- |- |- |- |- |- |- |- |- |- |- |- |-

        POR XMM0, XMM13              ; xmm0: R12|R13|R14|R6|R7|R8|R9|R10|R11|- |- |- |- |- |- |-
        POR XMM0, XMM9               ; xmm0: R12|R13|R14|R6|R7|R8|R9|R10|R11|R0|R1|R2|R3|R4|R5|-
        PSHUFB XMM0, [invertir]      ; xmm0: R0|R1|R2|R3|R4|R5|R6|R7|R8|R9|R10|R11|R12|R13|R14|-
    POP RBP
    RET
\end{verbatim}
\end{codesnippet}

\subsubsection{temp_aux:}

En la primer parte de \'esta funci\'on auxiliar usamos m\'ascaras para separar los colores y sumarlos entre si, dos pixeles al mismo tiempo.
Luego armamos para dividir este valor por tres, dando el promedio.
\begin{codesnippet}
\begin{verbatim}
temp_aux:
    PUSH RBP					
    MOV RBP, RSP
    PUSH R12
    PUSH R13
    
    XORPD XMM2, XMM2
    XORPD XMM3, XMM3
    XORPD XMM4, XMM4
    
    MOVDQA XMM7, [pixel0BGR]
    PSHUFB XMM0, XMM7		 			; ordeno en el registro los pixels de forma 0RGB (tengo 4 pixeles)
    MOVDQU XMM2, XMM0					; XMM2: R|0|G|0|B|0|0|0|R|0|G|0|B|0|0|0	
    PSHUFB XMM2, [soloPrimero]			; XMM2: R|0|0|0|0|0|0|0|R|0|0|0|0|0|0|0
    MOVDQU XMM3, XMM0					; XMM3: R|0|G|0|B|0|0|0|R|0|G|0|B|0|0|0 	
    PSRLQ XMM3, 16						; XMM3: G|0|B|0|0|0|0|0|G|0|B|0|0|0|0|0		
    PSHUFB XMM3, [soloPrimero]			; XMM3: G|0|0|0|0|0|0|0|G|0|0|0|0|0|0|0
    MOVDQU XMM4, XMM0					; XMM4: R|0|G|0|B|0|0|0|R|0|G|0|B|0|0|0 	
    PSRLQ XMM4, 32						; XMM4: B|0|0|0|0|0|0|0|B|0|0|0|0|0|0|0
        ; Ahora podemos hacer suma
    PADDQ XMM2, XMM3
    PADDQ XMM2, XMM4					; XMM2: qword[R+G+B] qword[R+G+B] 
    XORPD XMM5, XMM5
        ; suma / 3
    MOV R12, 3
    MOV R13, 1
    XORPD XMM0, XMM0
    XORPD XMM1, XMM1
    CVTSI2SS XMM0, R12
    CVTSI2SS XMM1, R13
        ; PARA EXTENDER EL 3 A LOS PACKS DEL XMM5
    XORPD XMM5, XMM5
    ADDSS XMM5, XMM1
    PSLLDQ XMM5, 4
    ADDSS XMM5, XMM0
    PSLLDQ XMM5, 4
    ADDSS XMM5, XMM1
    PSLLDQ XMM5, 4
    ADDSS XMM5, XMM0
    CVTDQ2PS XMM2, XMM2
    DIVPS XMM2, XMM5					; XMM2: prom (float)
    CVTTPS2DQ XMM2, XMM2
    CVTDQ2PS XMM2, XMM2
    MOVDQU XMM12, XMM2					; XMM12: prom (float)
    CVTTPS2DQ XMM12, XMM12
        ;XMM2: [T|0|0|0|0|0|0|0] [T|0|0|0|0|0|0|0]
\end{verbatim}
\end{codesnippet}

En esta segunda parte se elije mediante m\'ascaras qu\'e pixel ser\'a devuelto. 
\begin{codesnippet}
\begin{verbatim}
    XORPD XMM0, XMM0
        ; ponemos colores[0] en su lugar correspondiente	{128 + 4t, 0, 0}
    
    MOVDQU XMM10, [colores0] 			; XMM10: qword[128(+4T)] qword[128(+4T)] 
    
        ;XMM10: colores[0] nos falta agregar 4T
        ;XMM2: [T|0|0|0|0|0|0|0] [T|0|0|0|0|0|0|0]
    MOVDQU XMM6, XMM2
        ;XMM6: [T|0|0|0|0|0|0|0] [T|0|0|0|0|0|0|0]
    
        ;PARA MULTIPLICAR POR 4
    MOV R12, 4
    XORPD XMM14, XMM14
    MOVQ XMM14, R12
    XORPD XMM5, XMM5
    PADDQ XMM5, XMM14
    PSLLDQ XMM5, 8
    PADDQ XMM5, XMM14
    
    CVTDQ2PS XMM5, XMM5
    MULPS XMM6, XMM5	; XMM6: qword[4T|0|0|0|0|0|0|0] qword[4T|0|0|0|0|0|0|0]
    
    CVTTPS2DQ XMM6, XMM6
    
        ; A: 4T + 128
        ; XMM6: 	qword[4T|0|0|0|0|0|0|0] 		qword[4T|0|0|0|0|0|0|0]
        ; XMM10: 	qword[128(+4T)|0|0|0|0|0|0|0] 	qword[128(+4T)|0|0|0|0|0|0|0]
    PADDUSW XMM10, XMM6
    MOVDQU XMM6, XMM10
        ;XMM10 == XMM6 == qword[4T+128] qword[4T+128]
    XORPD XMM15, XMM15
    PUNPCKLQDQ XMM10, XMM15
    PUNPCKHQDQ XMM6, XMM15
        ;XMM10 == qword{[A|0|0|0|0|0|0|0] [0|0|0|0|0|0|0|0]}
        ;XMM6 ==  qword{[A|0|0|0|0|0|0|0] [0|0|0|0|0|0|0|0]}
        ;Lo voy a mirar de otra forma para empaquetar (puedo porque con 16bits 
		                                       ;me alcanza para representar A):
        ;XMM10 == dword{[A|0] [G|0] [B|0] [0|0] [0|0] [0|0] [0|0] [0|0]}
        ;XMM6 ==  dword{[A|0] [G|0] [B|0] [0|0] [0|0] [0|0] [0|0] [0|0]}
    PACKUSWB XMM10, XMM6
        ;XMM10 == dword{[AGB0][0000][AGB0][0000]}
        ;empaquete saturando word a byte para reducir el tamanio de A de 2 bytes a 1 byte
    PSHUFB XMM10, [shuffleCaverna2]
        ;XMM10 == dword{[AGB0][AGB0][0000][0000]}
    
    MOVDQU XMM1, [tresDos] 
    PCMPGTD XMM1, XMM12	
    PSHUFB XMM1, [shuffleCaverna2]
    PAND XMM1 , XMM10 					; estos pixeles le ponemos colores[0]
    XORPD XMM0, XMM0
    POR XMM0, XMM1						; ponemos colores0 en donde va en dst
	;///////////////////////////////////////////////////////////////////////////////

        ; ponemos colores[4] en su lugar correspondiente	{  0, 0, 1151 -4t}

        ;Codigo no copiado. Se repite el proceso que con [colores0]

\end{verbatim}
\end{codesnippet}

\begin{codesnippet}
\begin{verbatim}
	;///////////////////////////////////////////////////////////////////////////////

        ; ponemos colores[3] en su lugar correspondiente     {0, 895 - 4t, 255}

        ;Codigo no copiado. Se repite el proceso que con [colores0]
	;//////////////////////////////////////////////////////////////////////////////
        ; ponemos colores[2] en su lugar correspondiente		{639 -4t, 255, -384 + 4t},	

        ;Codigo no copiado. Se repite el proceso que con [colores0]
	;///////////////////////////////////////////////////////////////////////////////

        ; ponemos colores[1] en su lugar correspondiente		{255, -128 + 4t, 0}

        ;Codigo no copiado. Se repite el proceso que con [colores0]
	;///////////////////////////////////////////////////////////////////////////////	
    
    
        ;ahora en XMM2 tenemos dst con los colores finales
        ;sacamos los ceros para que queden pixeles de 3 bytes
        
        
    MOVDQU XMM10, [pixelFinal] ; [ 0 , 1 , 2 , 4, 5 , 6 , - , -, - , - , - , -, - , - , - , -]
    PSHUFB XMM0, XMM10
    
    POP R13
    POP R12
    POP RBP
    RET
\end{verbatim}
\end{codesnippet}

\subsection{Resultado de los experimentos}
\vspace*{0.3cm} \noindent
\textbf{Experimento 1}

\begin{figure}
  \begin{center}
	\includegraphics[scale=0.66]{imagenes/temp-lena.jpg}
	\caption{Lena}
	\label{Lena}
  \end{center}
\end{figure}

\begin{figure}
  \begin{center}
	\includegraphics[scale=0.66]{imagenes/temp-marilyn.jpg}
	\caption{Marilyn}
	\label{Marilyn}
  \end{center}
\end{figure}

Realizamos los tests armando un ciclo de 100000 ejecuciones del mismo c\'odigo con las mismas entradas, y a su vez ejecutamos estos tests 5 veces para cada entrada. \'Esto nos 
permiti\'o hacer promedios y descartar tests que daban muy lejos del valor medio.

Como muestran los gr\'aficos presentados, hay claras diferencias de velocidad (medida en cantidad de ciclos) entre uno y otro lenguaje. Tambi\'en notamos que no hay una gran 
variaci\'on de velocidad entre los distintos tamaños de las im\'agenes, as\'i como a veces las variaciones no son las esperadas. En este sentido, realizamos varias ejecuciones 
del TP2 con exactamente los mismos par\'ametros y vimos que variaban sin un patr\'on. Pensamos que si mejoraba con las sucesivas iteraciones podr\'ia ser 
producto de la acci\'on de la cache, pero como no fue as\'i vemos que tiene que ver con qu\'e tan ocupado est\'a el cpu. \'Esto no pudo ser confirmado ya que todos 
los tests se corrieron sin que ningun otro programa visible a nosotros est\'e corriendo.
\begin{figure}
  \begin{center}
	\includegraphics[scale=0.66]{imagenes/temp-lena-203.jpg}
	\caption{lena-203x203}
	\label{lena-203x203}
  \end{center}
\end{figure}

