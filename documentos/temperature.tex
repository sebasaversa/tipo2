\section{Filtro Temperature}
\subsection{Enunciado}

\subsection*{Filtro \textit{Temperature}}

  Programar el filtro \textit{Temperature} en lenguaje C y en en ASM haciendo uso de 
  las instrucciones vectoriales (\textbf{SSE}).


\vspace*{0.3cm} \noindent
\textbf{Experimento 1}

  Analizar cuales son las diferencias de performace entre las versiones de C y ASM. 
  Realizar gráficos que representen estas diferencias.
  
\subsection{An\'alisis previo}

El filtro temperature nuevamente presenta tres partes: el ciclo, la elecci\'on de c\'omo modificar los pixeles y la modificaci\'on propiamente dicha.

En \'este caso, la eleccin de los pixeles sale del promedio de los elementos RGB, y la modificaci\'on se realiza en base al promedio.

\subsection{Implementaci\'on en C}

En este caso no separamos la comparacion del pixel de su modificaci\'on porque cada pixel se modifica usando el valor del promedio de los valores RGB. El resto de la implementaci\'on es el ciclo del main que recorre toda la imagen.

\begin{codesnippet}
\begin{verbatim}
void temperatura(rgb_t* p_d, unsigned int r, unsigned int g, unsigned int b)
{
    unsigned int prom = (r + g + b) / 3;

    if 		(prom < 32)		{p_d->r = 0; 				p_d->g = 0; 				p_d->b = 128 + 4*prom;}
    else if (prom < 96)		{p_d->r = 0; 				p_d->g = -128 + 4*prom; 	p_d->b = 255;}
    else if (prom < 160)	{p_d->r = -384 + 4*prom;	p_d->g = 255; 				p_d->b = 639 - 4*prom;}
    else if (prom < 224)	{p_d->r = 255; 				p_d->g = 895 - 4*prom;	 	p_d->b = 0;}
    else 					{p_d->r = 1151 - 4*prom;	p_d->g = 0; 				p_d->b = 0;}
}
\end{verbatim}
\end{codesnippet}
\subsection{Implementaci\'on en assembler}
En este algoritmo tenemos ciertas partes muy parecidas a \textit{popart_asm}, como la sumatoria de los elementos RGB con \verb|PSHUFB| (para luego dividir por tres y obtener el promedio).

Sin embargo en este caso estas partes fueron encaradas de forma distinta. \\
Si comparamos los 2 algoritmos que realizan \verb|PSHUFB|, el algoritmo de la funci\'on \textit{popart} trabaja de a 4 pixels y la funci\'on \textit{temp_aux} de a 2 pixels.

Pero por ciclo, \textit{temperature_asm} procesa 5 pixeles en vez de los 4 que procesa \textit{popart_asm} por ciclo

\begin{codesnippet}
\begin{verbatim}
tempe:
    PUSH RBP
    MOV RBP, RSP
        ;XMM0: R|G|B|R|G|B|R|G|B|R|G|B|R|G|B|R coloco los primeros 16bytes de la imagen en XMM1
        ;XMM0: 0|1|2|3|4|5|6|7|8|9|10|11|12|13|14|15
        MOVDQU XMM8, XMM0
        CALL temp_aux
        MOVDQU XMM9, XMM0
        PSHUFB XMM8, [moverPixeles]  ; XMM8: 6 |7 |8 |9 |10|11|12|13|14|15|- |- |- |- |- |-
        PSHUFB XMM9, [moverPixeles2] ; XMM9: - |- |- |- |- |- |- |- |- |0 |1 |2 |3 |4 |5 |-
        MOVDQU XMM0, XMM8	
        CALL temp_aux
        MOVDQA XMM13, XMM0
        PSHUFB XMM13, [moverPixeles3]; xmm13: - |- |- |0 |1 |2 |3 |4 |5 |- |- |- |- | -|- |-

        PSHUFB XMM8, [moverPixeles1] ; xmm8:  12|13|14|- |- |- |- |- |- |- |- |- |- |- |- |-
        MOVDQU XMM0, XMM8
        CALL temp_aux                ; xmm0: ??
        PSHUFB XMM0, [unPixel]       ; xmm0:  0 |1 |2 |- |- |- |- |- |- |- |- |- |- |- |- |-

        POR XMM0, XMM13              ; xmm0: 0 |1 |2 |03|14|25|36|47|58|9 |10|11|12|13|14|15
        POR XMM0, XMM9               ; xmm0: 0 |1 |2 |03|14|25|36|47|58|09 |110|211|312|413|514|15
        PSHUFB XMM0, [invertir]      ; xmm0: 09|110|211|312|413|514|03|14|25|36|47|58|0 |1 |2 | -
    POP RBP
    RET
\end{verbatim}
\end{codesnippet}

\subsection{Resultado de los experimentos}
\vspace*{0.3cm} \noindent
\textbf{Experimento 1}


%\paragraph{\textbf{Titulo del parrafo} } Bla bla bla bla.
%Esto se muestra en la figura~\ref{nombreparareferenciar}.

%\begin{codesnippet}
%\begin{verbatim}

%struct Pepe {
%
%    ...
%
%};
%\end{verbatim}
%\end{codesnippet}

%\begin{figure}
%  \begin{center}
%	\includegraphics[scale=0.66]{imagenes/logouba.jpg}
%	\caption{Logo}
%	\label{Logo}
%  \end{center}
%\end{figure}


